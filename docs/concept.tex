\documentclass[paper=a4, fontsize=11pt]{scrartcl}
% \usepackage[T1]{fontenc}
\usepackage{fourier}
\usepackage[english]{babel}
\usepackage{amsmath,amsfonts,amsthm}
\usepackage{sectsty}
\usepackage{fancyhdr}
\usepackage[margin=0.5in]{geometry}

% \allsectionsfont{\centering \normalfont\scshape}  
\pagestyle{fancyplain}
\fancyhead{}
\fancyfoot[L]{}
\fancyfoot[C]{}
\fancyfoot[R]{\thepage}

\renewcommand{\headrulewidth}{0pt}
\renewcommand{\footrulewidth}{0pt}

\setlength{\headheight}{10pt}
\numberwithin{equation}{section}
\numberwithin{figure}{section}
\numberwithin{table}{section}
\setlength\parindent{0pt}

\newcommand{\state}[2]{\mathbb{S}_{#1}\left[\texttt{#2}\right]}
\newcommand{\horrule}[1]{\rule{\linewidth}{#1}} % Create horizontal rule command with 1 argument of height

\title{
    \normalfont \normalsize
    \textsc{Karlsruhe Institute of Technology} \\[25pt]
    \horrule{0.5pt} \\[0.4cm]
    \huge A basic stock exchange trading Bot \\
    \horrule{2pt} \\[0.5cm]
}
\author{Unknown6656}
\date{2017-11-11}

\begin{document}
    \maketitle
    %-------------------------------------------------------
    \section{Introduction}

	Notes:\\
	\textit{It shall be noted that the notation $ \mathbb{K}^\perp $ represents the set $ \mathbb{K} $ and the invalid or undefined state $ \perp $.}
	%-------------------------------------------------------
    \section{Concept}
    
    One first has to define a data type, which contains state information.\\
    The state contains the bot's budget volume (e.g. in \texttt{EUR}), the bot's asset count (meaning the number of shares that the bot posesses) and the last time the bot sold or bought shares.\\
    In this case we will define it as follows:
    \begin{align}
        \mathbb{S}_{i \in \mathbb{N}_0} := \mathbb{R}_0^+ \times \mathbb{N}_0 \times \left[a,b\right]^\perp \times \left[a,b\right]^\perp
    \end{align}
    The values $a$ and $b$ are positive integers which delimit the time range, in which the bot operates. One must also define the exchange rate as:
    \begin{align}
	    f \in \mathit{C'} \left( \left[a,b\right] \rightarrow \mathbb{R}_0^+ \right) \qquad
	    \text{with} \ a, b \in \mathbb{N}_0
    \end{align}
    One must define the following functions to represent a bot:
    \begin{align}
	    \texttt{init} & \ : \ () \longmapsto \mathbb{S}_a \\
	    \texttt{sell} & \ : \ \left[0..1\right] \times \mathbb{S}_{i-1} \rightarrow \mathbb{S}_i & (\mathbb{N}_0 \ni i > a) \\
	    \texttt{buy} & \ : \ \left[0..1\right] \times \mathbb{S}_{i-1} \rightarrow \mathbb{S}_i & (\mathbb{N}_0 \ni i > a) \\
	    \texttt{next} & \ : \ \mathbb{S}_{i-1} \rightarrow \mathbb{S}_i & (\mathbb{N}_0 \ni i > a) 
    \end{align}
    A so-called '\textit{risk-factor}' $r \in \left[0..1\right] $ and a temporal '\textit{lookback-factor}' $l \in \mathbb{R}^+ \setminus \{1,2\} $ should also be defined.
    \\
    In order to short-access the bot state's components, one also defines the following notation:
    \begin{itemize}
	    \item $ \state{i}{v} $ represents the bot's budget volume
	    \item $ \state{i}{c} $ represents the bot's asset count
	    \item $ \state{i}{b} $ represents the bot's last bought timestamp
	    \item $ \state{i}{s} $ represents the bot's last sold timestamp
    \end{itemize}
    \vspace{1cm}
	A concrete implementation could look as follows:
	\begin{align}
		\texttt{init} &\ : () \longmapsto \mathbb{S}_a = \left( \texttt{<budget>},\ 0,\ \perp,\ \perp \right) \\
		\texttt{buy} &\ : x, \mathbb{S}_i \longmapsto \left( \state{i-1}{v} - b f(i),\ \state{i-1}{c} + b,\ i,\ \state{i-1}{s} \right)
		 \quad & \text{with } b \ := \ \left\lfloor \frac{x\state{i-1}{v}}{f(i)} \right\rfloor
		\\
		\texttt{sell} &\ : x, \mathbb{S}_i \longmapsto \left( \state{i-1}{v} + s f(i),\ \state{i-1}{c} - s,\ \state{i-1}{b},\ i \right)
		\quad & \text{with } s \ := \ \left\lfloor \frac{x\state{i-1}{v}}{f(i)} \right\rfloor
		\\
		\texttt{next} &\ : \mathbb{S}_i \longmapsto
		\begin{cases}
			\mathbb{S}_{i-1} & \text{if } i < l + a \ \vee \ \left( f(i) > f(i-1) \ \wedge \ M_i(l) \leq r \right) \\
			\texttt{sell}(1,\ \mathbb{S}_i) & \text{if } i = b \\
			\\
			// todo
		\end{cases}
	\end{align}
\end{document}
